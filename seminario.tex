% Created 2022-05-30 Mon 11:21
% Intended LaTeX compiler: pdflatex
\documentclass[11pt]{article}
\usepackage[utf8]{inputenc}
\usepackage[T1]{fontenc}
\usepackage{graphicx}
\usepackage{longtable}
\usepackage{wrapfig}
\usepackage{rotating}
\usepackage[normalem]{ulem}
\usepackage{amsmath}
\usepackage{amssymb}
\usepackage{capt-of}
\usepackage{hyperref}
\author{Dasara Shullani Ph.D}
\date{\today}
\title{Needs more jpeg, Argenti edition}
\hypersetup{
 pdfauthor={Dasara Shullani Ph.D},
 pdftitle={Needs more jpeg, Argenti edition},
 pdfkeywords={},
 pdfsubject={},
 pdfcreator={Emacs 28.1 (Org mode 9.5.2)}, 
 pdflang={English}}
\begin{document}

\maketitle
\tableofcontents


\section{Signal processing appliced to Image Forensics}
\label{sec:orgc653bf9}
\begin{itemize}
\item Image forensics usata per capire se immagine è stata modificata.
\item Voilà, funziona la penna
\item Ci stanno mettendo un po' a iniziare, scusate il disturbo
\item Openboard permette di importare pdf, questa è un'ottima notizia
\item Il professore di automatica lo usa per le slide
\end{itemize}
\section{Ok, comincia}
\label{sec:org1e2feb7}
distinzione di segnali 1D, 2D, 3D

abbiamo visto i segnali 1D, ad esempio un segnale audio
\begin{itemize}
\item 1D, tipo audio
\item 2D, tipo foto
\item 3D, tipo video
\end{itemize}

Analisi seganli 2D per capire se un'immagine è stata modificata
Questa è una rappresentazione di un soggetto reale, la luce colpisce
un oggetto, la luce viene ricevuta da un sensore, questo segnale
luminoso viene discretizzato\par

Componente principale il sensore, casi principali
\begin{itemize}
\item CCD, light to photodiode
\item CMOS, light to photodiode ampilfier pair
\end{itemize}

Come si usa CFA per dare colore a immagine
(CFA coso verde verde rosso blu, chiedi a Captain Disillusion)
più verde perchè la banda del verde ricevuta meglio dai nostri occhi\par

campioni il rosso nei punti in cui la pattern è rossa, il resto del
rosso viene interpolato\par

Il passaggio successivo è un processo di compressione, viene
trasformato in bit (encoder) che usi per rivederla (decoder)

Algoritmi
\begin{itemize}
\item \textbf{Lossy}: immagine ricorstuita non identica (es. jpeg)
\item \textbf{Lossless}: decodifica rende immagine/ segnale identica (pensa
a un file .zip)
\end{itemize}

\section{Jpeg}
\label{sec:org8639e5c}

\subsection{Encoder}
\label{sec:org551354e}

\begin{itemize}
\item Trasformata DCT
\item Quantizzatore
\item Roba che mi sono perso
\end{itemize}

\subsection{Trasformata DCT}
\label{sec:org1ce04d8}

Discrete Cosine Transform, decompone un segnale attraverso le sue
armoniche coseno, somma delle armoniche pesate a dovere danno il
segnale originale.

In ambito immagine come si fa? Dividiamo in blocchi 8x8 non
sovrapposti, e fai la dft per tutti i blocchi
(chunk di un mondo minecraft, \textpm{})

\begin{itemize}
\item Basse frequenze, descrivono caratteristiche grossolane dell'immagine
\item Medie frequenze, un po' più fine
\item Alte frequenze, molto fini
\end{itemize}

Voglio un'immagine fedele alla realtà, ma che occupi il meno spazio
possibile

\subsection{Quantizzatore}
\label{sec:orgfe8a205}

La quantizzazione rimuove dettagli imprecettibili
lo stato di quantizzazine associa a ogni blocco 8x8 una matrice di
luminanza e crominanza (luminanza = intensità, crominanza =
colore), questo è il passo sostanziale di rimozione di dettagli\par

\subsubsection{I would like to interject for a moment}
\label{sec:org1aed204}

quelli sono uguali ai passi di quantizzazine, hai un passo \(\delta\) più
fine per i dettagli, in alte frequenze i passi sono più alti, in bassa
frequenza i passi sono minore, usare passi diversi permette di avere
più valori nulli, ce n'è uno per ogni valore della matrice

(se non si capisce, non ho capito un cazzo)


\subsection{Entropy coding}
\label{sec:orgfaa7ebe}
Claude shannon approva\par

Rappresentare nella maniera più efficiente possbilie, dire 10 0 e 1,
andiamo a fare codifica differenziale, poi boh, poi bah

\subsubsection{Primo passo, quantizzazione}
\label{sec:orgac224a0}

Quella cosa con l'albero per comprimere, robe comuni vanno a meno bit,
robe rare hanno più bit.

Huffman lookup table, quella cosa con l'albero

Stiamo comprimendo il comprimibile, per quanto possbile.

\section{Image forensics}
\label{sec:orgad03c9a}

Quest'immagine è stata manipolata?

le due macroaree sono
\begin{itemize}
\item St'immagine è stata modificata?
\item Da dove cazzo è venuta st'immagine? Da quale dispositivo?
\end{itemize}

\subsection{Verfica autenticità}
\label{sec:org93e11a4}
es. Jpeg non si allinea nella grigla 8x8, we call this a misalignment

ci sono due casi
\begin{itemize}
\item Doppio jpeg non allineato
\item Doppio jpeg allineato
\end{itemize}

\subsection{Demosaicatura}
\label{sec:org7ec94da}
molto visibile in un'immagine
se non si nota demosaicatura allora l'immagine potrebbe essere stata
modificata

\subsection{Identificazione sorgente}
\label{sec:org183c46d}
Identificazione del dispositivo che ha prodotto l'immagine

si studia l'impronta dell'immagine

un'immagine molto scura presenta rumore nelle parti scure
dell'immagine ad esempio, pensa a tutte le tue foto scacie di
tramonti.

vi sono rumori
\begin{itemize}
\item Shot noise
\item Pattern noise
\end{itemize}

Photo responsive Non uniformity noise
come, in fase di produzoine, si presentano artefatti di produzione.

Rumore estremamente presente in immagine naturale, e sopravvive molto
bene al processing.

Come si può usare questo rumore per identificare il sensore che ha
prodotto quest'immagine.

Il rumore è estremamente dipendente dall'immagine, viene ottenuto
facendo un denoising dell'immagine, e trovando la differenza tra
denoised e noised.

abbiamo bisogno di parecchie immagini prima di avere una idea completa
del rumore del sensore, meno immagini se hai tanti muri bianchi.

si testano i rumori in senso di correlazione col rumore nella foto per
vedere di capire da chi cazzo è arrivato sto rumore
\end{document}