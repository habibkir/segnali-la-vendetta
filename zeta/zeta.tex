% Created 2022-05-26 Thu 12:14
% Intended LaTeX compiler: pdflatex
\documentclass[11pt]{article}
\usepackage[utf8]{inputenc}
\usepackage[T1]{fontenc}
\usepackage{graphicx}
\usepackage{longtable}
\usepackage{wrapfig}
\usepackage{rotating}
\usepackage[normalem]{ulem}
\usepackage{amsmath}
\usepackage{amssymb}
\usepackage{capt-of}
\usepackage{hyperref}
\date{\today}
\title{}
\hypersetup{
 pdfauthor={},
 pdftitle={},
 pdfkeywords={},
 pdfsubject={},
 pdfcreator={Emacs 28.1 (Org mode 9.5.2)}, 
 pdflang={English}}
\begin{document}

\tableofcontents

\section{Definizione}
\label{sec:org22af5c0}
\subsection{Formula}
\label{sec:org8033981}
\[
X(z) = \sum_{n=-\infty}^{\infty} x[n] z^{-n}
\]
con \(z\) variabile complessa

\subsection{Rapporto con Fourier}
\label{sec:orga5ca6c9}
prendiamo la definizione di trasformata z e la definizione di Fourier
per sequenze

\begin{align*}
&X(z) = \sum_{n=-\infty}^{\infty} x[n] z^{-n} \\
&\overline{X}(F) = \sum_{n=-\infty}^{\infty} x[n] e^{-j2\pi Fn} \\
\end{align*}

le due formule sono abbastanza simili, entrambe hanno \(\sum x[n]\
qualcosa^{-n}\), nel caso di Fourier questo qualcosa = \(e^{j2\pi F}\),
nel caso della z questo può essere qualsiasi cosa, vediamo allora il
caso particolare in cui z ha la forma \(e^{j2\pi F}\).

\begin{align*}
  &X(z) \rvert _{z = e^{j2\pi F}} \\
= &(\sum_{n=-\infty}^{\infty} x[n] z^{-n})\rvert _{z = e^{j2\pi F}} \\
= & \sum_{n=-\infty}^{\infty} x[n] (e^{j2\pi F})^{-n} \\
= & \sum_{n=-\infty}^{\infty} x[n] e^{-j2\pi Fn} \\
= & \overline{X}(F)
\end{align*}

quindi \(X(z) \rvert _{z = e^{j2\pi F}}\) = \(\overline{X}(F)\), vale a
dire, la trasformata z equivale alla trasformata di Fourier per
sequenze nei punti della forma \(z = e^{j2\pi F}\), se lo riscriviamo
come \(z = 1 \times e^{j2\pi F}\) si può notare un po' meglio che questa
è una forma generica per indicare punti con modulo 1 e fase
arbitraria, in umanese vuol dire che è un punto sul cerchio unitario.
(se \(F\) arbitrario \(\in (-\frac{1}{2},\frac{1}{2})\) allora \(2\pi F\)
arbitrario \(\in (-\pi\) e \(\pi\)))

\section{Esempii}
\label{sec:orgf9d06fe}

\subsection{Sequenza finita}
\label{sec:orgb663558}
\subsection{Monolatera destra}
\label{sec:org16c3ed6}
si prenda la sequenza
\[
x[n] = 2^n u[n]
\]
allora
\begin{align*}
& X(z) = \sum_{n=-\infty}^{\infty} x[n] z^{-n} \\
& X(z) = \sum_{n=-\infty}^{\infty} x[n] = 2^n u[n] z^{-n} \\
\end{align*}

per il gradino
\begin{align*}
& X(z) = \sum_{n=0}^{\infty} x[n] = 2^n z^{-n} \\
& X(z) = \sum_{n=0}^{\infty} x[n] = (2z^{-1})^{-1} \\
\end{align*}

essendo questa diventata una serie geometrica, si ricordi intanto che
\[
f(q) = \sum_{n=0}^{\infty} q^n \Rightarrow
f(q) = \begin{cases}
\frac{1}{1-q} & \text{se } \lvert q \rvert < 0 \\
\text{non converge} & \text{se } \lvert q \rvert \geq 0
\end{cases}
\]

\subsection{Monolatera sinistra}
\label{sec:org1042e7d}

\subsection{Importanza della RoC}
\label{sec:org08327c5}
\section{Teoremi}
\label{sec:org857a110}
\subsection{Linearità}
\label{sec:orgd28098e}

\subsection{Ritardo}
\label{sec:orgfb79288}

\subsection{Inversione temporale}
\label{sec:org261b1d8}

\subsection{Derivazione in z}
\label{sec:org3f7cda7}

L'ipotesi è che

\begin{align*}
&x[n] \iff X(z) \\
&n x[n] \iff -z \frac{dX(z)}{dz}
\end{align*}

la dimostrazione è che se
\[
X(z) = \sum_{n=-\infty}^{\infty} x[n] z^{-n}
\]

allora facendo la derivata a entrabmi i lati ottieni
\[
\frac{dX(z)}{dz} = \frac{d}{dz}(\sum_{n=-\infty}^{\infty} x[n] z^{-n})
\]

sia la derivata che la sommatoria sono lineari, quindi puoi fare
\[
\frac{dX(z)}{dz} = \sum_{n=-\infty}^{\infty} \frac{d}{dz}(x[n] z^{-n})
\]

\$x[n] non dipende da \(z\), quindi si porta fuori dalla derivata e
\begin{align*}
&\frac{dX(z)}{dz} = \sum_{n=-\infty}^{\infty} \frac{d}{dz}(x[n] z^{-n}) \\
&\frac{dX(z)}{dz} = \sum_{n=-\infty}^{\infty} x[n] \frac{dz^{-n}}{dz} \\
&\frac{dX(z)}{dz} = \sum_{n=-\infty}^{\infty} x[n] -n z^{-n-1} \\
&\frac{dX(z)}{dz} = \sum_{n=-\infty}^{\infty} x[n] -n z^{-n} z^{-1}
\end{align*}

\(z^{-1}\) non dipende da n quindi lo portiamo fuori insieme al segno
\(-\) del \(-n\) introdotto dalla derivazione

\begin{align*}
&\frac{dX(z)}{dz} = -z^{-1} \sum_{n=-\infty}^{\infty} x[n] n z^{-n} \\
&-z \frac{dX(z)}{dz} = \sum_{n=-\infty}^{\infty} x[n] n z^{-n} \\
\end{align*}

si nota \footnote{come se qualcuno andasse effettivamente a cercarsi ste
cose} che la parte destra dell'equazione corrisponde a \(\mathcal{Z}\{n
x[n]\}\), da ciò si ottiene la tesi.

\[
\mathcal{Z}\{n x[n]\} = -z \frac{dX(z)}{dz}
\]

\subsection{Convoluzione}
\label{sec:org77917d9}
\end{document}