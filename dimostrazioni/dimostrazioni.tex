% Created 2022-06-04 Sat 13:03
% Intended LaTeX compiler: pdflatex
\documentclass[11pt]{article}
\usepackage[utf8]{inputenc}
\usepackage[T1]{fontenc}
\usepackage{graphicx}
\usepackage{longtable}
\usepackage{wrapfig}
\usepackage{rotating}
\usepackage[normalem]{ulem}
\usepackage{amsmath}
\usepackage{amssymb}
\usepackage{capt-of}
\usepackage{hyperref}
\date{\today}
\title{}
\hypersetup{
 pdfauthor={},
 pdftitle={},
 pdfkeywords={},
 pdfsubject={},
 pdfcreator={Emacs 28.1 (Org mode 9.5.2)}, 
 pdflang={English}}
\begin{document}

\tableofcontents

\section{Segnali TD}
\label{sec:org2399046}
\begin{itemize}
\item Campionamento
\item Interpolazione cardinale(?)
\end{itemize}

\section{Quantizzazione}
\label{sec:org4f0a955}
\begin{itemize}
\item Ipotesi del rumore, bla bla bla è bianco
\item Potenza dell'errore, bla bla bla \(\frac{\Delta ^2}{2}\)
\end{itemize}

\section{Trasformate}
\label{sec:orgc4070cb}
\subsection{Fourier per sequenze}
\label{sec:orgdc31610}
\begin{itemize}
\item Linearità
\item Ritardo
\item Modulazione
\item Coniugazione
\item Inversione temporale
\item Convoluzione
\item Prodotto
\item Parseval
\item Incremento
\item Definizione processo WSS
\item Cazzatine su densità spettrale di potenza
\end{itemize}
\subsection{Trasformata Z}
\label{sec:orgbd345cf}
\begin{itemize}
\item Linearità
\item Ritardo
\item \(y[n] = a^n x[n]\)
\item Coniugazione
\item Inversione temporale
\item Derivazione
\item Convoluzione
\item RoC sequenza limitata
\item RoC di monolatere destre, sinistre
\item RoC bilatera
\end{itemize}
\subsection{DFT}
\label{sec:orgba6c988}
\begin{itemize}
\item Linearità
\item \(\Phi [m]\)
\item Ricordati che è circolare
\item Ritardo
\item Modulazione
\item Inversione temporale
\item Coniugazione
\item Parseval
\end{itemize}

\section{Esempii di quesiti chiesti}
\label{sec:org1291aa3}
\begin{itemize}
\item Proprietà processo aleatorio per essere stazionario
\item Rapporto tra fourier di \(y[n]\), sequenza somma di \(x[n]\), e
\(x[n]\) stessa
\item Scrivi espressione di funzione ottenuta con interpolazione a
mantenimento di \(x[n]\)
\item Simmetria di Fourier per le sequenze reali
\item Data una quantizzazione uniforme, relazione tra \(x[n]\) e
\(\hat{x[n]}\)
\item Enunciare ipotesi assunte per errore di quantizzazione
\item\relax [DATA EXPUNGED]
\end{itemize}

\section{Esempii di dimostrazioni chiesti}
\label{sec:org4090b56}
\begin{itemize}
\item Enunciare il teorema del prodotto di Fourier per sequenze
\item Dimostrazione relazione tra Fourier di \(x(t)\) e Fourier di
\(x[n]\)
\item Dimostrare che dall'antitrasformata discreta discende la
trasformata discreta di Fourier
\item Enunciare il teorema della convoluzione della DFT
\end{itemize}
\end{document}