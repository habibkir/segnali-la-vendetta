% Created 2022-06-02 Thu 17:13
% Intended LaTeX compiler: pdflatex
\documentclass[11pt]{article}
\usepackage[utf8]{inputenc}
\usepackage[T1]{fontenc}
\usepackage{graphicx}
\usepackage{longtable}
\usepackage{wrapfig}
\usepackage{rotating}
\usepackage[normalem]{ulem}
\usepackage{amsmath}
\usepackage{amssymb}
\usepackage{capt-of}
\usepackage{hyperref}
\usepackage{mathcal}
\date{\today}
\title{}
\hypersetup{
 pdfauthor={},
 pdftitle={},
 pdfkeywords={},
 pdfsubject={},
 pdfcreator={Emacs 28.1 (Org mode 9.5.2)}, 
 pdflang={English}}
\begin{document}

\tableofcontents


\section{Proprietà dei sistemi}
\label{sec:orge8038e5}
\subsection{Lineare}
\label{sec:org6bcf15d}
\subsection{Tempo invarianza}
\label{sec:org7811ddb}
\subsection{Senza memoria}
\label{sec:org9483087}
\subsection{Stabile}
\label{sec:orgeb76735}
Si definisce stabile un sistema con stabilità \textbf{BIBO} (Bouded Input
Bouded Output), vale a dire un sistema \(y[n] = \mathcal{T}\{x[n]\}\)
tale che \(\forall\) \(x[n]\) limitato \(\Rightarrow\) \(y[n]\) è limitato, si
dimostra che un sistema LTI è BIBO stabile se e solo se h[n] è
assolutamente sommabile, quindi 

\[
\mathcal{T}\{\dot\} \text{ è BIBO } \iff
\sum_{n=-\infty}^{\infty} \lvert h[n] \rvert < \infty
\]

dimostrazoine:

Avanti : \(\sum \lvert h[n] \rvert < \infty \Rightarrow\) BIBO
\begin{verbatim}
Errore: Dimostrazione non ancora inserita, siete pregati di andare a fanculo
\end{verbatim}

Indietro : BIBO \(\Rightarrow \sum \lvert h[n] \rvert < \infty\)
La dimostrazoine è per assurdo, avrà quindi la forma
\[
\Rightarrow \sum \lvert h[n] \rvert = \infty \Rightarrow
\text{\textbf{NON} BIBO}
\]

iniziamo da questa definizione del cazzo:
\[
y[n] = \sum_{k=-\infty}^{\infty} h[k] x[n-k]
\]

vogliamo un input \(x[n]\) che faccia "risonare" \(h[n]\) e che la faccia
esplodere malamente, per facilità di non dover scrivere \(n\) in tutte
le formule facciamo che vogliamo farlo esplodere in \(n=0\).

per far esplodere \(h[n]\) vediamo quali caratteristiche di \(h[n]\)
rendono buoni esplodsivi

sappiamo che \(h[n]\) non è assolutamente sommabile (ed è l'unica cosa
che sappiamo di \(h[n]\) da ste ipotesi quindi non abbiamo molta altre
scelta)

\subsection{Causale}
\label{sec:orge635064}

Un sistema causale rispetta il principio del non viaggiare nel tempo,
il valore di \(y[n]\) non può quindi dipendere da valori di \(x[n]\) quali
\$x[n+1], x[n+2], x[n+3]\ldots{}\$, v
\end{document}