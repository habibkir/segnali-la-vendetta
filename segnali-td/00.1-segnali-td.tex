% Created 2022-05-19 Thu 12:06
% Intended LaTeX compiler: pdflatex
\documentclass[11pt]{article}
\usepackage[utf8]{inputenc}
\usepackage[T1]{fontenc}
\usepackage{graphicx}
\usepackage{longtable}
\usepackage{wrapfig}
\usepackage{rotating}
\usepackage[normalem]{ulem}
\usepackage{amsmath}
\usepackage{amssymb}
\usepackage{capt-of}
\usepackage{hyperref}
\date{\today}
\title{}
\hypersetup{
 pdfauthor={},
 pdftitle={},
 pdfkeywords={},
 pdfsubject={},
 pdfcreator={Emacs 28.1 (Org mode 9.5.2)}, 
 pdflang={English}}
\begin{document}

\tableofcontents

\section{Segnali tempo discreto}
\label{sec:org8f3b7ba}

\subsection{Definizione}
\label{sec:org2428fab}
Un segnale tempo discreto è una sequenza con indice temporale, puoi
ottenerlo:
\begin{itemize}
\item Misurando una quantitià discreta nel tempo (n. di auto/ora
a un tornello autostradale
\item Prendendo un sengale tempo continuo e \textbf{campionandolo}
\end{itemize}

Ai fini di questo corso possiamo ignorare l'esistenza del primo caso

\subsection{Campionamento}
\label{sec:org63301ee}

Voglio mettere un file audio in un computer, quinid ho sto segnale
x(t) e devo campionarlo, vorrei quindi prendere una sequenza x[0],
x[1], x[2], x[3]\ldots{} di valori assunti da questo segnale. Un modo per
farlo è definire x[n] come
\begin{align*}
&x[0] = x[0T] \\
&x[1] = x[1T] \\
&x[2] = x[2T] \\
&x[3] = x[3T] \\
&x[4] = x[4T] \\
&x[5] = x[5T] \\
&... 
&x[n] = x[nT] \\
\end{align*}

con T pari a un intervallo di tempo costante, quindi ogni tot fisso di
secondi prendiamo un campione,se per esempio T = 0.2 secondi per
esempio, il che ci darebbe:
\begin{align*}
&x[0] = x[0.0s] \\
&x[1] = x[0.2s] \\
&x[2] = x[0.4s] \\
&x[3] = x[0.6s] \\
&x[4] = x[0.8s] \\
&x[5] = x[1.0s] \\
&... 
&x[n] = x[(n*0.2)s] \\
\end{align*}

\subsubsection{Claude Shannon è un figo della madonna}
\label{sec:org834d089}

Un modo molto usato in questo corso per prendere un campione di
qualcosa è moltiplicarlo per una delta di Dirac, per avere 
il caso sopra con x[n] = x(nT) vorremo isolare x(nT), possiamo isolare
x(nT) facendo x(t)\texttimes{} \(\delta\) (t-nT), rocordandoci che la \(\delta\)
estrae il punto dove il suo argomento fa 0 (estrae il punto dove
esplode).\par 

per avere l'intera sequenza vorremo isolare x(0T), x(1T), x(2T),
etc\ldots{} , per fare questo dovremo fare il prodotto con \(\delta\) (t-0T),
\(\delta\) (t-1T), \(\delta\) (t-2T), etc\ldots{} , e avere tutti i campioni messi
insieme. Questo si ottiene moltiplicando per il \emph{pettine di
campionamento di Dirac}.

\[
p(t) = \sum_{n=-infty}^{\infty} \delta (t-nT)
\]
\end{document}