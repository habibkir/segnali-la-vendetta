% Created 2022-06-03 Fri 12:48
% Intended LaTeX compiler: pdflatex
\documentclass[11pt]{article}
\usepackage[utf8]{inputenc}
\usepackage[T1]{fontenc}
\usepackage{graphicx}
\usepackage{longtable}
\usepackage{wrapfig}
\usepackage{rotating}
\usepackage[normalem]{ulem}
\usepackage{amsmath}
\usepackage{amssymb}
\usepackage{capt-of}
\usepackage{hyperref}
\date{\today}
\title{}
\hypersetup{
 pdfauthor={},
 pdftitle={},
 pdfkeywords={},
 pdfsubject={},
 pdfcreator={Emacs 28.1 (Org mode 9.5.2)}, 
 pdflang={English}}
\begin{document}

\tableofcontents

\section{Definizione}
\label{sec:org6b80e5d}
è un campionamento della trasformata di fourier per sequenze

se \(x[n]\) fosse infinita prenderemmo campioni infinitamente fitti, non
propio un'ottima idea
data la trasformata per sequenze, \(X(F)\) è periodica, dato che è un
campionamento, \(X[k]\) è periodica

\[
X[k] = X(F) \rvert _{F=\frac{k}{N}} \]

la DFT si applica a sequenze periodiche o di durata finita.
\[
x[n] \Rightarrow X(F) = \sum_{n = -\infty}^{\infty}
x[n] e^{-j2\pi Fn} \]

visto che il segnale è finito, per quanto detto sopra
\begin{align*}
&x[n] \Rightarrow X(F) = \sum_{n = 0}^{N-1} x[n] e^{-j2\pi Fn} \\
&X[k] = X(F) \rvert _{F=\frac{k}{N}}
= \sum_{n = 0}^{N-1} x[n] e^{-j2\pi \frac{k}{N}n}
\end{align*}

\section{C'è anche l'antitrasformata}
\label{sec:org238b8f1}

\subsection{Quest'altra sequenza che useremo poi}
\label{sec:org6ff8113}

Iniziamo enunciando sto risultato parziale, prendetela per buona, tra
un paio di sezioni si capisce perché dovrebbe avere senso\footnote{nei
modi e nei limiti delle capacità esplicative dell' Argetnti}
introdurre questa cosa.
si definisca la serie
\[
\Phi [m] = \sum_{k = 0}^{N-1} e^{-j2\pi \frac{k}{N}m}
\]

si dimostra che
\[
\Phi[m] = \begin{cases}
N & \text{ se } m = 0, m = \pm N, m = \pm 2N... \\
0 & \text{ altrimenti }
\end{cases}
\]

mi faceva fatica scrivere i calcoli in \LaTeX\ in tempo reale, vedi gli
appunti del deste.
c'è qualche \(e^{j2\pi \times \text{intero}} = 1\), poi fa una serie
geometrica parziale

\subsection{Formula dell'antitrasformata}
\label{sec:org1153d81}

Senza provare a far capire perché si definisce in questo modo, ecco la
formula dell'antitrasformata, per spiegare senza
spiegare\footnote{cit. Marco de Stefano}

\[ x[n] = \frac{1}{N} \sum_{k=0}^{N-1} X[k] e^{+j2\pi \frac{k}{N}n} \]

facendo qualche sostituzione del cazzo possiamo dimostrare che
funziona

una sottospecie di processo per dimostrare ciò è sotto riportato,
partendo da
\[ x[n] = \frac{1}{N} \sum_{k=0}^{N-1} X[k] e{+j2\pi \frac{k}{N}n} \]

riscriviamo \(X[k]\) come trasformata di \(x[n]\), visto che la lettera
\(n\) è già occupata useremo la lettera \(r\) per la sommatoria interna
\[ x[n] = \frac{1}{N} \sum_{k=0}^{N-1}
(\sum_{r = 0}^{N-1} x[r] e^{-j 2\pi \frac{k}{N}r})
e^{+j2\pi \frac{k}{N}n} \]

poi succede un miracolo, e torna
\[ x[n] \]


\section{Periodicità}
\label{sec:org2a51363}

\subsection{Anche x[n] è periodica}
\label{sec:org67d464e}
se metti \(x[n + N]\) nella formula di \(x[n]\) ottenuta a partire da
\(X[k]\) ti ritorna \(x[n]\), quindi è periodica, se la cosa magari ti interessava.


\section{Propietà della trasformata}
\label{sec:org42dc8d2}

\subsection{Linearità}
\label{sec:org381e902}

Grazialcazzo

\subsection{Teorema del ritardo}
\label{sec:org464bb9a}

Visto che stiamo lavorando con sequenze periodiche al ritardo \$x[n-n\textsubscript{0}]
corrisponderà un \textit{ritardo deluxe} che viene definito come
\textit{traslazione circolare}.

Tanto la tesi è la stessa
\begin{align*}
& x[n] \iff X[k] \\
& x[n-n_0] \iff X[n] e^{-j2\pi \frac{k}{N} n_0}
\end{align*}

\subsection{Traslazione in frequenza}
\label{sec:org8610a2c}

Qui non prova neanche a dimostrarla, ecco la tesi, au revoir
\[ x[n] e^{j\frac{2\pi k_0}{N}n} \iff X[k - k_0] \]

\subsection{Inversione temporale}
\label{sec:org87abc05}

abbiamo \(y[n] = x[-n]\)
\begin{align*}
& x[n] \iff X[k] \\
& y[n] \iff \text{?}
\end{align*}

qualcosa, mi sono perso mentre lo stavate scrivendo

\subsection{Teorema di parceval}
\label{sec:orgce559cb}

facendo il solito cazzo di risultato intermedio

\[
\sum_{n=0}^{N-1} x[n]y^*[n] =
\frac{1}{N} \sum_{k=0}^{N-1}X[k]Y^*[k]
\]

metti \(y[n] = x[n]\)


\begin{align*}
& \sum_{n=0}^{N-1} x[n]x^*[n] =
\frac{1}{N} \sum_{k=0}^{N-1}X[k]X^*[k] \\
& \sum_{n=0}^{N-1} \lvert x[n] \rvert ^2 =
\frac{1}{N} \sum_{k=0}^{N-1} \lvert X[k] \rvert ^2 \\
\end{align*}
\end{document}
